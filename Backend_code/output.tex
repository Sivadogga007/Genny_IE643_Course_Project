\documentclass[preprint,authoryear]{elsarticle}
\usepackage{amssymb}
\journal{}%Mathematical Biosciences}
\usepackage{graphicx}
\usepackage{amsmath,amssymb,amsfonts,latexsym}
\usepackage{color}
\usepackage{eucal} % Caligraphic Euler fonts: \mathcal{}
\usepackage{xspace}
\usepackage{hyperref}
\usepackage{bm}
\usepackage[title]{appendix}
\newtheorem{proposition}{Proposition}
\newtheorem{proof}{Proof}
\usepackage[round]{natbib}
\usepackage{algorithm}
\usepackage[noend]{algpseudocode}
\DeclareMathOperator*{\argmax}{argmax} % No space, limits underneath in displays
\DeclareMathOperator*{\argmin}{argmin} % No space, limits underneath in displays

\begin{document}

\title{Sample Title for the Document}
\author{Your Name}
\address{Your Institution}

\begin{abstract}
This is a sample abstract for the LaTeX document. It summarizes the content of the paper.
\end{abstract}

\begin{keyword}
Keyword1 \sep Keyword2 \sep Keyword3
\end{keyword}

\section{Introduction}
This is the introduction section of the document. Here you can introduce your topic.

\section{Main Content}
\subsection{Subsection Example}
This is an example of a subsection. You can include equations, such as:

\begin{equation}
E = mc^2
\end{equation}

\section{Conclusion}
This is the conclusion of the document.

\begin{appendices}
\section{Appendix Example}
This is an example of an appendix.
\end{appendices}

\end{document}

    